% -*- root: 00-main.tex -*-
\section{Introduction}
\label{sec:introduction}

\IEEEPARstart{F}{usion} of information from different images is a difficult
  task in image processing, especially when the scale of the imaged structures
  is in the range of the image resolution or below.
In the case of the human brain, the cortex has a typical thickness of up to 5 mm
  \citep{fischl_measuring_2000}.
Moreover, as the cortex is densely folded, points located afar in the topological
  manifold of the brain's surface can be very close in the corresponding 3D space.
Consequently, accuracy and precision are very difficult to achieve, but necessary properties
  of registration algorithms.

In this paper we propose a novel registration framework to simultaneously
solving the segmentation, distortion and cortical parcellation challenges,
by exploiting as strong shape-prior the detailed morphology extracted
from high-resolution and anatomically correct \gls{mri}.
Indeed, hereafter
we assume this segmentation problem in anatomical images is reliably and
accurately solved with readily available tools (e.g.
\citep{fischl_freesurfer_2012}).
After global alignment with \gls{t1} using existing approaches, the remaining
spatial mismatch between anatomical and diffusion space is due to susceptibility
distortions.
Finally, we need to establish precise spatial correspondence between the
surfaces in both spaces, including the tangential direction for parcellation.
Therefore, we can reduce the problem to finding the differences of spatial
distortion in between anatomical and \gls{dwi} space.
We thus reformulate the segmentation problem as an inverse problem, where we
seek for an underlying deformation field (the distortion) mapping
from the structural space into the diffusion space, such that the structural
contours segment optimally the \gls{dwi} data.
In the process, the one-to-one
correspondence between the contours in both spaces is guaranteed, and projection
of parcellisation to \gls{dwi} space is implicit and consistent.

We test our proposed joint segmentation-registration model on two different
synthetic examples.
The first example is a scalar sulcus model, where the
\gls{csf}-\gls{gm} boundary particularly suffers from \gls{pve} and can only be
segmented correctly thanks to the shape prior and its coupling with the inner,
\gls{gm}-\gls{wm} boundary through the imposed deformation field regularity.
The second case deals with more realistic \gls{dwi} data stemming from
phantom simulations of a simplistic brain data.
Again, we show that the
proposed model successfully segments the \gls{dwi} data based on two derived
scalar features, namely \gls{fa} and \gls{md}, while establishing an estimate
of the dense distortion field.

The rest of this paper is organized as follows.
First, in \autoref{sec:methods}
we introduce our proposed model for joint multivariate segmentation-registration.
Then we provide a more detailed description of the data and experimental setup in
\autoref{sec:experiments}.
We present results in \autoref{sec:results} and conclude
in \autoref{sec:conclusion}.