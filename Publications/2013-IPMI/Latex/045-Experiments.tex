\section{Data and experiments}
\label{sec:experiments}
%
\subsection{Simulated datasets}
%
As described in \autoref{sec:introduction}, the general situation in
the connectivity pipelines consists on having 
a reliable segmentation obtained from the high resolution \ac{t1} 
reference image. Therefore, a precise location of the tissue interfaces
of interest is available in a reference space. Given that there is no 
interest on the anatomical reference segmentation,
we directly obtained the shape priors from the models. \\

On the other hand, the target \ac{dwi} data is characterized by its
low resolution (typically around $2.2x2.2x3mm^3$). Depending on the
posterior reconstruction methodology and the angular resolution
intended, the \ac{dwi} raw data has to be processed in order to
extract the information in a manageable manner. Particularly, we
will use the \ac{fa} and \ac{md} maps for convenience.
Whereas \ac{fa} describes the \emph{shape} of diffusion, 
the \ac{md} depicts the \emph{manitude} of the process. 
There exist two main reasons to justify their choice. 
First, they are well-understood and standardized in clinical routine.
Second, together they contain most of the information that is
usually extracted from the \ac{dwi}-derived scalar maps. \\

In order that demonstrating the functionality of the proposed
methodology and characterize its possibilities, we developed two
synthetic models and simulated their corresponding \ac{dwi}
raw signal as described in \citep{tuch_q-ball_2004}. 
(HERE WE NEED A GOOD DESCRIPTION OF THE DATA, directions, resolution, etc)
. The first model consists of several spherical shapes emulating
the different brain tissues (see \autoref{fig:fa}, first row). 
The second model is based on the BrainWeb dataset.
We reconstructed the \ac{dwi} signal with standard procedures to approximate the
environment to the real one at maximum. \\

%
\subsection{Experiment}
%
For both models, we created manually a sound distortion visually similar
to real \ac{epi} distortions. We interpolated the distortion to a 
dense deformation field, necessary for warping the raw \ac{dwi} simulated
data. Once the signal was deformed, we proceeded to reconstruct the
\ac{dti} and subsequently obtained the scalars of interest (\ac{fa}, \ac{md}) and estimated their parameters on the model.\\

We evaluate the performance of our methodology to estimate the deformation
field and compare it to the original synthetic deformation field applied to the data.
