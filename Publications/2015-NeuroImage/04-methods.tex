% -*- root: 00-main.tex -*-
\section{Methods}\label{sec:regseg-methods}

\subsection{Registration framework and segmentation model}\label{sec:regseg-methods_map}
Let $\gammaset_R = \{\Gamma_m: m \in \mathbb{N}, m \leq N_S\}$ be the set of $N_S$ surfaces
  extracted from the undistorted \gls*{t1} image (the reference space $R$).
We reformulate the segmentation of the distorted \gls*{dmri} images (the moving space $M$)
  as a registration problem where we search for an underlying deformation field $U$ such that
  the structures in $R$ defined by $\gammaset_R$ align optimally with their corresponding
  structures in $M$:

  \begin{align}
  U\colon R \subset \mathbb{R}^n &\to M \subset \mathbb{R}^n \notag\\
  \vec{r} &\mapsto \vec{r}' =\vec{r}+\vec{u}(\vec{r}),
  \label{eq:regseg-transform}
  \end{align}
  where $\vec{r}$ denotes a position in $R$, $\vec{r}'$ is
  its corresponding location in $M$, and $n$ denotes the dimensionality of images.
Finally, $\vec{u} = \vec{u}(\vec{r})$ is the displacement of every point with respect
  to the reference domain.
The general overview of how the surfaces interact with the registration framework
  is presented in \autoref{fig:regseg-method}.

\paragraph*{Cost-function derivation}
In a Bayesian framework for registration \citep{wyatt_map_2003,pohl_bayesian_2006,gass_simultaneous_2014},
  the mappings $U$ in \eqref{eq:regseg-transform} are
  evaluated based on their posterior probability given the observed data
  $M$.
Let $\omegaset = \{\Omega_l: l \in \mathbb{N}, l \leq N_L\}$ be the set of $N_L$ competing regions in
  which $M$ is partitioned by the projection of $\gammaset_R$.
Using Bayes' rule, the posterior likelihood is computed as:

  \begin{equation}
  P(U \mid M, \omegaset ) = \frac{P(M \mid U,\omegaset )\, P(U)}{P(M)},
  \label{eq:regseg-bayes_rule}
  \end{equation}
  where $P(M \mid U,\omegaset)$ is the data likelihood.
Since $\omegaset$ are mapped by $U$, we simplify
  $P(U, \omegaset) = P(U) \implies P(M \mid U,\omegaset) = P(M \mid U)$.
The best estimate $\hat{U}$ then satisfies the maximum a posteriori criterion
  and it aligns $\gammaset_R$ into $M$.
First, we assume independence between \revcomment[R\#3-C10]{voxel}s, and thus we break down the
  global data likelihood into a product of \revcomment[R\#3-C10]{voxel}-wise conditional probabilities:

  \begin{equation}
  P(M \mid U) = \underset{l}{\prod} \underset{\vec{r}'\in \Omega_l}{\prod}
    P\left( \vec{f}' \mid U \right),
  \label{eq:regseg-bayes_aposteriori}
  \end{equation}
  where $\vec{f}' = M(\vec{r}')$ is the feature vector at the displaced
  position $\vec{r}'$ \eqref{eq:regseg-transform} in the moving image.
For convenience and because it has been shown to be an appropriate approximation
  \citep{leemput_automated_1999,cuadra_comparison_2005}, we introduce two assumptions for each
  region $\Omega_l$:
  1) the features are i.i.d.; and 2) they can be modeled by multivariate normal
  distributions 
  $\mathcal{N} (\vec{f}' \mid \boldsymbol{\mu}_l, \boldsymbol{\Sigma}_{l} )$, with parameters
  $\lbrace \boldsymbol{\mu}_l, \boldsymbol{\Sigma}_{l} \rbrace$
  for each region $\Omega_l$ \citep{esteban_mbis_2014}:

  \begin{align}
  P( M \mid U) &= \underset{l}{\prod} \underset{\vec{r}' \in \Omega_l}{\prod}
  \mathcal{N} (\vec{f}' \mid \boldsymbol{\mu}_l, \boldsymbol{\Sigma}_{l} )
  \ifthenelse{\boolean{review}}{=}{= \notag\\ &=}
  \underset{l}{\prod} \underset{\vec{r}' \in \Omega_l}{\prod} \frac{1}{ \sqrt{(2\pi)^{C}\,\left|\boldsymbol{\Sigma}_{l}\right|}}\,{e^{\left(-\frac{1}{2}
  \mdist{f'}{l} \right)}},
  \label{eq:regseg-pdf}
  \end{align}%
  using $\mdist{f'}{l}$ to denote the squared \emph{Mahalanobis distance} of $\vec{f}'$ with respect
  to the descriptors of region $l$ as
  $\mdist{f'}{l} = (\vec{f}' - \boldsymbol{\mu}_l)^T \, {\boldsymbol{\Sigma}_l}^{-1} \, (\vec{f}' - \boldsymbol{\mu}_l)$.
$C$ is the number of channels comprised in the image $M$.
Even though the features being segmented are not generally i.i.d., the 
  spatial interdependency of voxels is implicitly supported by the piecewise smooth
  partition of the space \omegaset{}.
In fact, the projection of $\gammaset_R$ onto $M$ is an implicit segmentation model, for which
  the covariance matrix $\boldsymbol{\Sigma}_l$ of each region is minimized.
The \autoref{fig:regseg-model} shows how the joint distribution of 
  the input images is approximated with a mixture of multivariate normal distributions,
  and this minimization is illustrated for the segmentation of the \gls*{fa} and the \gls*{adc} maps
  of one subject.


\begin{figure}
  \includegraphics[width=\linewidth]{figure01}
  \caption{The interfacing surfaces $\Gamma_m$ between the competing \glspl{roi} $\Omega_l$,
  play the role of active contours which drive the registration process.
  They evolve iteratively along the normal $\hat{\mathbf{n}}_i$ of the surface at each vertex
    $\vec{v}_i$ of the mesh.
  The gradient speeds $\bar{s}_i$ drive registration, which are computed as the disparity of the data
    energies with respect to the two limiting regions of $M(\vec{v}_i)$, the features of the image
    $M$ in the location of vertex $\vec{v}_i$.
  The computation of shape-gradients is developed in \ref{app:shape_gradients}.
  In this figure, the $\bar{s}_1$ derived from \autoref{eq:regseg-shape_gradient_disc2} is 
    written in the lower box, with $\Omega_{wm}$ in the inner limiting region, $\Omega_{gm}$
    the outer region, and $w_{0,1}$ is the relative area associated with vertex $\vec{v}_1$
    with respect to the total area of surface $\Gamma_0$.
  }\label{fig:regseg-method}
\end{figure}

\begin{figure*}
% \includestandalone[width=\linewidth]{figure02-model}
\begin{minipage}{\textwidth}
\includegraphics[width=\linewidth]{figure02}\\%
\vskip-2.0em
\includegraphics[width=\linewidth]{legend}%
\end{minipage}
% \includegraphics[width=\linewidth]{figure04B}
% \includegraphics[width=\linewidth]{figure04C}
\caption{%
Evolution of the segmentation model defined by the homogeneous regions $\Omega_l$,
  for one real dataset.
Panel (A, left) shows the joint distribution of the \gls*{fa} and \gls*{adc} conditioned to
  the segmentation \omegaset{} defined by the surfaces $\gammaset_R$ extracted from
  the \gls*{t1} image.
The plot was generated for reference using undistorted diffusion data, and
  therefore, $\gammaset_R$ is aligned with the \gls*{fa} and the \gls*{adc}.
The problem arises when the diffusion data present deformation, and the
  contours $\gammaset_R$ do not fit within the data (A, center).
After registration with \regseg{}, the contours are mapped onto the
  diffusion data (A, right), and the joint density plot is closer to the reference
  situation.
In panel (B), the three plots in (A) are decomposed tissue-wise.
Using filled contours, the bivariate distribution of each tissue is highlighted in its designated color,
  and represented over the remaining tissues (in gray colors).
To help assessment, dashed contours in black-to-white colors represent
  the corresponding distribution in the reference plot.
The registration process optimizes the segmentation model of \regseg{},
  and thus, the distribution of each region after registration is located
  closer to that corresponding in the reference situation, the shape of the distribution
  is more similar to the reference, and their spread is also reduced.
The effects of optimization are more noticeable on the \gls*{gm} ($\Omega_\text{GM}$) and
  the \gls*{wm} ($\Omega_\text{WM}$).
Particularly, the \gls*{wm} typically shows a bimodal distribution when the contours
  \gammaset{} do not fit the data.
% The left plot in each panel represents a kernel density estimate of the location and spread of
%   each region $\Omega_l$ in the bivariate feature space of $M$, which comprises
%   the \gls*{fa} and \gls*{adc} maps.
% At the top and the right margins, the marginal distributions of $\Omega_l$ on the \gls*{fa} and
%   \gls*{adc} are respectively displayed.
% On the right side, the sampled distribution of each region is shown in small scatter plot, in contrast to
%   the complete bivariate data.
% A) shows the initialization of the model, when the anatomically-correct surfaces are not
%   mapped into the \gls*{dmri} space.
% The panel shows the resulting conditional distributions of segmenting the (distorted) \gls*{fa} and
%   \gls*{adc} maps with the original (undistorted) surfaces.
% Panel B shows the equivalent information, using the surfaces that are perfectly mapped into the
%   distorted \gls*{dmri} space.
% Since the surfaces are now aligned in $M$, the spreads of the conditional distributions for each region are
%   reduced with respect to those in A.
% Panel C, after \regseg{} registration, shows how the covariance matrices $\boldsymbol{\Sigma}_l$
%   are minimized with respect to panel A, and the resulting segmentation is close
%   to the theoretical situation presented in panel B.}
}\label{fig:regseg-model}
\end{figure*}

\paragraph*{Regularization}
The smoothness of the resulting displacement field is induced
  by a Thikonov regularization prior:

  \begin{align}
  P(U) = \underset{\vec{r}}{\prod}\, p(\vec{u}) &=
  \underset{\vec{r}}{\prod}\, p_0(\vec{u}) \, p_1(\vec{u}), \text{ with} \\
  p_0(\vec{u}) &= \mathcal{N}( \vec{u} \mid 0, \mathbf{A}^{-1}), \notag\\
  p_1(\vec{u}) &= \mathcal{N}(  \nabla \vec{u} \mid 0, \mathbf{B}^{-1}),
  \label{eq:regseg-thikonov}
  \end{align}
 which requires that the distortion and its gradient have zero mean,
   and variance governed by the matrices $\mathbf{A}$ and $\mathbf{B}$.
Therefore, $\mathbf{A}$ and $\mathbf{B}$ are tensors that modulate the regularization, and
  produce deformations with preferential directions.
Finally, the maximum a posteriori problem is adapted to a variational problem where we search for
  the minimum of an energy functional by applying $E(\vec{u}) = -\log \{P( M \mid U) \, P(U)\}$:

  \begin{align}
  E(\vec{u}) &= -\log \underset{l}{\prod}
  \underset{\vec{r}' \in \Omega_l}{\prod}
  \mathcal{N} \left( \vec{f}' \mid \boldsymbol{\mu}_l, \boldsymbol{\Sigma}_l \right)\,p_0( \vec{u})\,p_1( \vec{u}) =
\notag\\ &= - \underset{l}{\sum} \int_{\Omega_l} \Big\{ \log \left[ \mathcal{N} \left( \vec{f}' \mid \boldsymbol{\mu}_l, \boldsymbol{\Sigma}_l \right) \right] \, + \log \left[ p_0( \vec{u})\,p_1( \vec{u}) \right] \Big\} \, d\vec{r}'
   = \notag\\ &=
  \const + \underset{l}{\sum} \Big\{ \int_{\Omega_l} \mdist{f'}{l} \,d\vec{r} \Big\} \,
  + \int_{\Omega} \frac12 \left[ \vec{u}^T \mathbf{A} \vec{u} + (\nabla \vec{u})^T \mathbf{B} (\nabla \vec{u}) \right] \,d\vec{r}'.
  \label{eq:regseg-energy}
  \end{align}%
This expression is the dual of the Mumford-Shah functional that corresponds
  to the framework of \acrlong*{acwe} \citep{chan_active_2001}
  with the anisotropic regularization term of \cite{nagel_investigation_1986}.


\subsection{Numerical Implementation}
\label{sec:regseg-numerical_implementation}

\paragraph*{Deformation model}\label{sec:regseg-deformation_model}
Since the vertices of the surfaces $\{\vec{v}_i: \vec{v}_i \subset \gammaset \}_{i=1 \ldots N_V}$
  are probably located off-grid, it is necessary to derive $\vec{u}_i = \vec{u}(\vec{v}_i)$ from a discrete set of parameters
  $\{\vec{u}_k\}_{k=1 \ldots K}$.
Densification is achieved using a set of associated basis functions $\psi_k$ \eqref{eq:regseg-nodes_tfm}.
In our implementation, $\psi_k$ is a tensor-product B-spline kernel of degree three.

  \begin{equation}
  \vec{v}_i' = \vec{v}_i + \vec{u}_i = \vec{v}_i + \sum_k \psi_k(\vec{r}) \: \vec{u}_k.
  \label{eq:regseg-nodes_tfm}
  \end{equation}


\paragraph*{Optimization}
\label{sec:regseg-gradient_descent}
To find the minimum of the energy functional \eqref{eq:regseg-energy},
  we propose a gradient-descent approach with respect to the underlying
  deformation field using the following \gls*{pde}:

  \begin{equation}
  \frac{\partial \vec{u}(\vec{r},t)}{\partial t} \propto - \frac{\partial E(\vec{u})}{\partial \vec{u}_k},
  \label{eq:regseg-general_gradient_descent}
  \end{equation}
  where $t$ is an artificial time parameter of the contour
  evolution and $\vec{u}_k$ are the parameters that support the estimate
  $\hat{U}$ of the transformation at the current time point.
Let us assume that the preferential directions of
  the displacement are aligned with the imaging axes to simplify \eqref{eq:regseg-energy} as expression 
  \eqref{eq:regseg-app_energy} in \ref{app:reg_term}, and thus to compute its
  derivative \eqref{eq:regseg-general_gradient_descent}:

  \begin{align}
  \frac{\partial E(\vec{u})}{\partial \vec{u}_k} &=
  \frac{ \partial }{\partial \vec{u}_k} \Big\{
  \underset{l}{\sum} \big[ \int_{\Omega_l} \mdist{f'}{l} \,d\vec{r}' \big] \,
  + \int_{\Omega} \frac12 [ \boldsymbol{\alpha} \cdot \vec{u}^{\circ2}
  + \boldsymbol{\beta} \cdot (\nabla \vec{u})^{\circ2} ] \,d\vec{r}'
  \Big\},
  \label{eq:regseg-gradient_descent}
  \end{align}
  where $\vec{u}^{\circ2} = \vec{u}^T \cdot \vec{u}$,
  and $\{\boldsymbol{\alpha}, \boldsymbol{\beta}\}$ are the expected variances along
  the imaging axes of the displacement field and its gradient, respectively.
Then, the data and regularization terms are split and discretized to compute their
  derivatives.
The derivative of the data term is computed using explicit shape gradients (see \ref{app:shape_gradients}),
  which finally lead to obtain vertex-wise speeds of the gradient $\bar{s}_i$ as
  illustrated in \autoref{fig:regseg-method}.
\revcomment[R\#1-C4]{%
The shape gradient contributions $\vec{g}_k$ on the field coefficients $\vec{u}_k$
  can then be computed }%
using the expressions \eqref{eq:regseg-gradient_wshape} and 
  \eqref{eq:regseg-shape_gradient_final} given in \ref{app:shape_gradients}.
Finally, introducing the analytical derivative of the regularization term,
  the \autoref{eq:regseg-gradient_descent} is reformulated as:

  \begin{align}
  \frac{\partial E(\vec{u})}{\partial \vec{u}_k} &=
  \vec{g}_k  + \boldsymbol{\alpha} \cdot \vec{u}_k - \boldsymbol{\beta} \cdot (\Delta \vec{u}_k),
  \label{eq:regseg-final_gradient}
  \end{align}
where $\vec{g}_k$ are the shape-gradient contributions on the coefficients $\vec{u}_k$ of the B-spline grid.
Finally, to descend this gradient, we establish a semi-implicit Euler scheme
  \citepalias[see][section S1.3]{esteban_useful_2015},
  with a step size parameter $\delta$, which we solve in the spectral domain as follows:

  \begin{align}
  \vec{u}_k^{t+1} = \mathcal{F}^{-1}\left\{ \frac{\mathcal{F}\{\delta^{-1} \, \vec{u}_k^t - \vec{g}_k\} }
                    {\mathcal{F}\{(\delta^{-1} + \boldsymbol{\alpha})\, I - \boldsymbol{\beta}\Delta\}} \right\},
  \label{eq:regseg-update_equation}
  \end{align}
  where $I$ denotes the identity operator.


\paragraph*{Implementation details and convergence}%
\label{sec:regseg-conv_report}
The \regseg{} tool includes a multiresolution strategy on the free-form deformation field.
Registration pyramids are created by setting the spacing between the control points of the B-spline basis
  functions for each level of the multiresolution strategy.
As a rule of thumb, for a $\delta =$ 1.0, both $\boldsymbol{\alpha}$
  and $\boldsymbol{\beta}$ will typically be in the range [0.0, 1.0].
The parameters used ($\delta$, $\boldsymbol{\alpha}$, $\boldsymbol{\beta}$, the B-spline grid resolutions,
  and target image smoothing), the implementation details, and other features such as the sparse matrix
  approach to fast interpolation are discussed in the \citetalias[][section S1]{esteban_useful_2015}.

\subsection{Evaluation protocol}\label{sec:regseg-evaluation_protocol}
In order to assess the performance of \regseg{}, we defined the following general
  evaluation protocol:
1) Extract the set of undistorted surfaces $\gammaset_R$;
2) Compute a ground-truth field of displacements $U_\text{true}$, which is applied to
  generate warped images ($M$) for segmentation;
3) Execute \regseg{} with $\gammaset_R$ and use the warped data as inputs; and
4) Perform a visual assessment and compute the error metrics.

A first proof of concept is introduced to demonstrate \regseg{} in digital phantoms
  with simple geometries, using $U_\text{true}$ without directional restrictions.
Then, \regseg{} is evaluated in a framework using undistorted \gls*{dmri} datasets,
  and $U_\text{true}$ is derived from the corresponding inhomogeneity fieldmap of
  the subject.
Therefore, the deformation field is nonzero only in the \acrfull*{pe} axis,
  and reproduces a real \gls*{epi} distortion.
The adaptation of the evaluation protocol to the simulated phantoms and the real data is
  explained in the following sections.

\subsection{Simulated phantoms}\label{sec:regseg-digital_phantoms}
The workflow required to simulate the digital phantoms and to assess the performance of
  \regseg{} with them is presented in \autoref{fig:regseg-evphantoms}.
A set of four binary objects (i.e. ``box'', ``ball'', ``L'',
  and ``gyrus'') was generated by combining the binarization of
  analytical shapes and mathematical morphology.
The reference surfaces $\gammaset_R$ were extracted from the binary shapes
  using \emph{FreeSurfer} tools \citep{fischl_freesurfer_2012}.
The ground-truth distortion was generated using a chain of two displacement
  fields supported by grids of B-spline basis functions.
The coefficients of the basis functions were generated randomly for
  both levels in their three dimensions.
The three components of the displacements $\vec{u} = (u_d)$
  were bounded above by 40\% of the separation between the control points
  at each level to obtain diffeomorphic transforms
  after concatenation \citep{rueckert_diffeomorphic_2006}.
The first deformation field was applied to generate large warpings
  with control points separated by 50.50 mm in the three dimensions
  ($u_d\leq$ 20.20 mm).
With the second warping, we aimed to obtain a field with smoothness
  close to that found in a typical distortion field of \gls*{dmri} data
  \citep{irfanoglu_susceptibility_2011}.
Therefore, the control points were separated by 25.25 mm ($u_d\leq$ 10.10 mm).
After generating the ground-truth deformation, the original surfaces
  were warped by interpolating the displacement field at each vertex.
The warped surfaces $\gammaset_\text{true}$ were binarized to generate tissue fractions
  at low (\isores[mm]{2.0}) and high (\isores[mm]{1.0}) resolutions.
Using an \acrshort*{mri} simulator \citep{caruyer_phantomas_2014}, we synthesized
  \gls*{t1} (TE/TR= 10/1500 ms) and \gls*{t2} images (TE/TR= 90/5000 ms), which
  corresponded to each phantom type, with each at two resolutions
  (1.0 mm and 2.0 mm isotropic).
The field of view at both resolutions was \isores[mm]{100}.
Next, \regseg{} was applied to map $\gammaset_R$ onto the warped phantoms to
  obtain the registered surfaces ($\hat{\gammaset}_\text{test}$).
To quantify the misregistration error, we computed the Hausdorff distance between
 $\hat{\gammaset}_\text{test}$ and $\gammaset_\text{true}$ using \citep{commandeur_vtk_2011}.
In total, 1200 experiments (four phantom types $\times$ 150 random warpings $\times$ two resolutions) were
  performed according to the workflow illustrated in \autoref{fig:regseg-evphantoms}.

\begin{figure*}
\includegraphics[width=\linewidth]{figure03}
\caption{Evaluation of \regseg{} using phantom data according to the following instrumental workflow.
  1) The reference surfaces $\gammaset_R$ are triangle meshes extracted from the four binary shapes (i.e., ``box'', ``ball'', ``L'', ``gyrus'').
  2) A ground-truth displacement field was generated as described in \autoref{sec:regseg-digital_phantoms}, and applied to warp
      $\gammaset_R$, thereby obtaining $\gammaset_\text{true}$.
  3) After being warped, $\gammaset_\text{true}$ were projected onto the corresponding discrete 3D volume and downsampled to create partial volume effects at two resolutions,
     i.e., \isores[mm]{2.0} and \isores[mm]{1.0}, thereby producing sets of tissue fractions maps.
  4) The tissue fractions were fed into an \acrshort*{mri} simulator, which generated \acrfull*{t1} and \acrfull*{t2} -like images at the
     two possible resolutions.
  5) The \regseg{} tool was applied using the warped test images as multispectral moving images and $\gammaset_R$ as shape priors.
  6) The agreement between the surfaces fitted by \regseg{} ($\hat{\gammaset}_\text{test}$) and $\gammaset_\text{true}$ were assessed
     visually using automatically generated visual reports and quantitatively with the Hausdorff distance between the
     corresponding surfaces.}\label{fig:regseg-evphantoms}
\end{figure*}

\paragraph*{Segmentation model and settings}
The segmentation model of the phantoms is implicitly defined:
  all phantoms comprehend an inner surface enclosing a uniform \gls*{wm}-like region,
  and an outer surface wrapping a \gls*{gm}-like layer.
The outside is filled with uniform background (see \autoref{fig:regseg-evphantoms}).
All the experimental settings used for the phantoms are made available in
  a unique configuration file\footnote{\url{https://github.com/oesteban/RegSeg/blob/master/Scripts/pyacwereg/data/regseg_default.json}}.

\subsection{Real datasets} \label{sec:regseg-human_connectome}
The experimental framework for the real datasets is presented in \autoref{fig:regseg-evworkflows},
which extends our previous evaluation \citep{esteban_simulationbased_2014} of distortions
  using \gls*{dmri} phantoms.

\begin{figure*}
\includegraphics[width=\linewidth]{figure04}
\caption{Experimental workflow employed to process real data from the \acrfull*{hcp}.
  1) $\gammaset_R$ were extracted from the anatomical reference (\gls*{t1} image).
  2) For use as the ground truth, we generated a plausible synthetic distortion $U_\text{true}$
    from the fieldmap with \eqref{eq:regseg-fieldmap}.
  3) The \gls*{dmri} data were warped using $U_\text{true}$ to reproduce the effects of real
    susceptibility-derived distortions.
  Target diffusion scalars (\gls*{fa} and \gls*{adc}) were computed with the distorted data and
    stacked to feed the multivariate input required by \regseg{}.
  4) The method was run to obtain $U_\text{test} = \hat{U}_\text{true}$, i.e., the estimate of
    the ground-truth deformation.
  5) The results were evaluated visually and quantitatively.
  The arrows point to edges in the target images (light-yellow arrows for \gls*{fa}, blue for \gls*{adc}) that
    should be aligned with a surface, showing how distortion limits the direct mapping from the structural space
    in which the contours are defined.}\label{fig:regseg-evworkflows}
\end{figure*}


\paragraph*{Data}
To evaluate \regseg{} using real \gls*{dmri} data obtained from human brains,
  we collected 16 subjects from the \revcomment[R\#3-C9]{Q3 Release of the}
  \gls*{hcp} database.
The original acquisitions are released within ``unprocessed'' packages, whereas
  the ``minimally preprocessed'' packages contain the corresponding images after
  some processing (correction for several artifacts, brain-extraction, spatial
  normalization, etc.).
We refer the reader to \citet{essen_human_2012} for exact details of the acquisition
  parameters and \citet{glasser_minimal_2013} for the preprocessing issues.
These datasets comprise a large set of images, including \gls*{t1}, \gls*{t2}, and
  multi-shell \gls*{dmri} images.
Since we obtained the \gls*{dmri} data from the minimally preprocessed package, these
  images are corrected for \gls*{epi} distortions and spatially normalized in
  \gls*{t1} space.
\revcomment[R\#3-C9]{In the Q3 Release of the \gls*{hcp}, the \gls*{dmri} session includes
  six runs, two runs for each of three different gradient tables, and each table is acquired
  once with right-to-left and left-to-right encoding polarities, respectively.
Then, the diffusion datasets with opposed polarities are corrected for susceptibility
  distortions using the \emph{TOPUP} tool \citep{andersson_how_2003}
  before producing the released ``minimally preprocessed'' data.}.

\paragraph*{Segmentation model}
Based on our experience % \citep{esteban_brain_2012}
  and previous studies \citep{ennis_orthogonal_2006},
  we defined the moving image as a stack of the \gls*{fa} and \gls*{adc} maps derived
  from \gls*{dmri} data.
After evaluating several alternative models, we empirically defined a partition \omegaset{}
  according to the following six regions:
  1) thalamus ($\Omega_{Tha}$);
  2) ventricular system and deep \gls*{gm} structures ($\Omega_{VdGM}$);
  3) cerebral \gls*{wm} ($\Omega_{WM}$);
  4) brain stem and cerebellar \gls*{wm} ($\Omega_{bst}$);
  5) cerebellar \gls*{gm} ($\Omega_{cbGM}$); and
  6) cortical \gls*{gm} ($\Omega_{GM}$).
Using tools in \emph{FreeSurfer} and appropriate selections of labels in the
  \emph{aparc} segmentation released with the \gls*{hcp} data, we extracted the $\gammaset_R$ set for the
  reference surfaces.
The segmentation model corresponding to this partition is shown in \autoref{fig:regseg-model}
  and discussed in greater detail in the \citetalias[section S4]{esteban_useful_2015}.

\paragraph*{Ground-truth generation}
Realistic deformation was achieved by generating displacement fields that satisfy the theoretical
  properties of distortion.
The displacements along the \gls*{pe} axis of the \gls*{dmri} image are related to the local
  deviation of the field $\Delta B_0(\vec{r})$ from its nominal value $B_0$  \citep{jezzard_correction_1995},
  as follows:

  \begin{equation}
  u_\text{PE} = \frac{\gamma \, T_{acq}\, s_\text{PE}}{2\pi}\Delta B_0(\vec{r})\text{ [mm]},
  \label{eq:regseg-fieldmap}
  \end{equation}
where $\gamma$ is the gyromagnetic ratio, $T_{acq}$ is the readout time, and
  $s_\text{PE}$ is the \revcomment[R\#3-C10]{voxel} size along \gls*{pe}.
Certain \acrshort*{mri} sequences are designed to estimate $\Delta B_0$, thereby obtaining
  the so-called fieldmap.
We derived the deformation $U_\text{true}$ from the fieldmap image released with
  the corresponding packages of each dataset in the \gls*{hcp}.
The fieldmap was unwrapped\footnote{fieldmaps are phase maps, which are intrinsically clipped in the interval
  of $[-\pi, \pi)$ [rads] or [rads/s].} and smoothed before applying \eqref{eq:regseg-fieldmap}.
Next, the original \gls*{dmri} was warped using the resulting displacement field and fed into
  a pipeline to process the corresponding \gls*{dti}, thereby computing the derived scalars of
  interest (\gls*{fa} and \gls*{adc}) using \emph{MRtrix} \citep{tournier_mrtrix_2012}.

\paragraph*{Metric assessment}
Initially, we investigated the appropriateness of the segmentation model.
For five test datasets, we uniformly sampled the space of distortions
  $\hat{U} = \epsilon \cdot U_\text{true} = \vec{r} + \epsilon \, u_\text{PE}$
  (with $\epsilon \in [-1.1, 1.1]$ and $u_\text{PE}$ from \eqref{eq:regseg-fieldmap}),
  and we evaluated the data term of the cost function \eqref{eq:regseg-energy}.
The minimum of the cost function (\autoref{sec:regseg-methods_map}) was consistently located at
  $\epsilon=0.0$ (the ground-truth) for all of the cases \citepalias[figure S2]{esteban_useful_2015}.

\paragraph*{Settings} %
\Regseg{} accepts an affine mapping from surface-space to the \gls*{dmri} data as initialization.
However, the images provided by the \gls*{hcp} are already spatially normalized.
Therefore, the initial estimation of distortion is zero in this experiment.
Since the distortion $U_\text{true}$ is aligned along the \gls*{pe} direction ($y$-axis in our settings),
  \regseg{} was configured to allow nonzero displacements only on that corresponding direction.
For the experiments on real data, \regseg{} established a multi-resolution pyramid of
  B-spline functions, with control points distributed on grids of the following spacings:
  40$\times$100$\times$40 [mm] for the first (coarser) level, \isores[mm]{30} for the second level,
  and 20$\times$30$\times$10 [mm] in the third level.
Only the first and second levels included Gaussian smoothing of the target image ($\sigma$ = [2.0, 0.5] mm,
  respectively).
The actual choices of the parameter settings are publicly distributed with the source code for the
  experiments\footnote{\url{https://github.com/oesteban/RegSeg/tree/master/Scripts/pyacwereg/data/regseg_hcp.json}}.
These settings were obtained manually, driven by the feedback obtained from the post-registration
  convergence reports \citepalias[like that found in][section S1.3]{esteban_useful_2015}.
We released \regseg{} along with the tool to generate such convergence reports.

\paragraph*{Cross-comparison} %
A dual workflow to the general evaluation used for \regseg{} (\autoref{fig:regseg-evworkflows}),
  was employed to integrate the alternate \gls*{t2b} registration scheme.
We reproduced the solution and settings provided with \emph{ExploreDTI}
  \citep{leemans_exploredti_2009}, which is a widely used toolkit for tractography analysis of
  \gls*{dti}.
\emph{ExploreDTI} internally employs \emph{elastix} \citep{klein_elastix_2010} to
  perform registration.
The deformation field is correspondingly restricted to the \gls*{pe} direction.
The settings file for \emph{elastix} is also
  available\footnote{\url{https://github.com/oesteban/RegSeg/blob/master/Scripts/pyacwereg/data/t2b_elastix_y.txt}}.
\revcomment[R\#3-C7\\R\#3-C8]{%
In this registration scheme, the \gls*{t2} image is the \emph{reference} and
  the \lowb{} plays the role of \emph{moving} image.
Therefore, the transform is defined in the coordinate system of the \gls*{t2} image,
  and for each point in this space provides the location of the corresponding feature
  in the \lowb{} image.
Since the surfaces are defined in the \gls*{t2} --\emph{reference}-- space, their coordinates
  can be mapped to the \lowb{} space using this transform, obtaining the 
  \emph{distorted surfaces} corresponding to the \lowb{}-to-\gls*{t2} registration.}


\paragraph*{Error measurement}\label{sec:regseg-experiments_evaluation}
Distortion only occurs along the \gls*{pe} axis of the image, so we computed the
  \gls*{swindex} as the area-weighted distance between the corresponding vertices of
  $\gammaset_\text{true}$ and their estimate obtained by the method under the test $\hat{\gammaset}_\text{test}$:

  \begin{equation}
  sWI = \frac{1}{\sum_i a_i} \sum\limits_i^{N_V} a_i\,\|
  \vec{v}_i - \hat{\vec{v}}_i \|,
  \label{eq:regseg-swindex}
  \end{equation}
  where $\vec{v}_i \subset \gammaset_\text{true}$ are the locations of the total $N_V$ vertices,
  $a_i$ is the area corresponding to each vertex $\vec{v}_i$,
  and $\hat{\vec{v}}_i \subset \hat{\gammaset}_\text{test}$ are the recovered locations
  that correspond to $\vec{v}_i$.
In practice, we only report the \gls*{swindex} for three surfaces (\{$\Gamma_{VdGM}, \Gamma_{WM}, \Gamma_{pial}$\})
  of crucial interest in whole-brain tractography.
\revcomment[R\#3-C7\\R\#3-C8]{%
The \gls*{swindex} is always computed on the \gls*{dmri} space.}