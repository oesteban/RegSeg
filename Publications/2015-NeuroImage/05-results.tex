% -*- root: 00-main.tex -*-
\section{Results}
\label{sec:results}

\subsection{Proof of concept on digital phantoms}
\label{sec:results_phantom}

\begin{figure*}
    \centering
    \resizebox{\textwidth}{!}{%
      \begin{tikzpicture}
        \node[inner sep=10pt, draw=black!40](fig03b) at (0,7.9)
          {\hspace*{20pt}\includegraphics[width=\linewidth]{figures/figure03-b}};
        \node[inner sep=10pt, draw=black!40](fig03c) at (0,0)
          {\hspace*{20pt}\includegraphics[width=\linewidth]{figures/figure03-c}};
        \node[circle, text=black!75] at (-9.4,10.6) {\Huge \textbf{A}};
        \node[circle, text=black!75] at (-9.4,3.5) {\Huge \textbf{B}};
      \end{tikzpicture}
    }%
  \caption{A. Visual assessment of the results on the low resolution sets:
    ``gyrus'' (top-left), ``L'' (top-right), ``ball'' (bottom-left),
    and ``box'' at (bottom-right).
  In yellow color, the recovered contours after registration are represented.
  Our method showed high accuracy, as it demonstrates the almost exact location of the registered
    contours with respect to their ground truth position depicted in green.
  Partial volume effect turns segmentation of the sulci a challenging problem with voxel-wise
    clustering methods, but it is successfully segmented with our method.
  B. Quantitative evaluation of registration error in terms of average Hausdorff distance of
    surfaces at low (left) and high (right) resolutions, demonstrating that the error is
    consistently below the voxel size.
    }\label{fig:phantom}
\end{figure*}
A total of 1200 experiments (4 phantom types $\times$ 150 random warpings $\times$ 2 resolutions) were
  run using the workflow given in \autoref{fig:evphantoms}.
For each experiment, the misregistration error was measured using the Hausdorff distance
  (see \autoref{sec:experiments_evaluation}) between the theoretical $\gammaset_\text{true}$ and
  the estimation done by \emph{regseg} ($\hat{\gammaset}_{test}$).
The results showed a consistent and high accuracy, below the image resolution.
\autoref{fig:phantom} (block C) presents the violin plots by model type corresponding
  to the two sets of resolutions of generated phantoms.
In order to relate average misregistration error to the resolution of the moving image,
  we proceeded as follows.
First, we confirmed that the vertex-wise error distributions were skewed using a Shapiro-Wilk test of
  normality.
All the distributions of errors under test (4 phantom types $\times$ 2 resolutions) resulted
  non-normal with $p<0.001$.
Consequently, we used the non-parametric Wilcoxon signed-rank test along with Bonferroni
  correction for multiple comparisons ($N=150$, for each phantom type).
Averaged errors resulted significantly lower than voxel size with $p < (0.001 / 150)$
  for all the tests (4 phantom types $\times$ 2 resolutions).
Since statistical tests may not be conclusive enough, we also computed the confidence intervals,
  reported in \autoref{tab:ci_phantom}.


\begin{table}
		\centering
		\footnotesize
    \tabcolsep=0.1cm
    \begin{tabular}{lccccc}
    Res.   & ``gyrus'' & ``ball''  & ``box''   & ``L''     & Aggreg.    \\\hline
    1.0mm  & 0.18-0.38 & 0.31-0.45 & 0.34-0.42 & 0.34-0.40 & 0.34-0.38  \\
    2.0mm  & 0.59-0.60 & 0.65-0.76 & 0.68-0.71 & 0.67-0.77 & 0.64-0.66  \\
    \hline
    \end{tabular}
    \caption{The vertex-wise Haussdorf distance between the ground-truth surfaces and their
    corresponding estimation with \emph{regseg} presented distributions with 95\% CI below
    the half-voxel size in all the phantom types.
    CIs were computed using bootstrapping of 10$^4$ samples, and median as location statistic.}\label{tab:ci_phantom}
\end{table}

\subsection{Evaluation on real datasets and cross-comparison}\label{sec:results_hcp}
%
Finally, we compared the performance of \emph{regseg} and the standard \gls*{t2b}
  method.
Visual assessment of the 16 cases is included in the \suppl{section S5}.
In \autoref{fig:results_real}, box A, the visual report for one subject is supplied.
We computed the \gls*{swindex} \eqref{eq:swindex} of every surface after registration,
  for both \emph{regseg} and the \gls*{t2b} methods.
Finally, to statistically compare the results, we performed Kruskal-Wallis H-tests
  (a non-parametric alternative to ANOVA) on the warping indices for three specific 
  surfaces of interest selected in \autoref{sec:experiments_evaluation}.
All the statistical tests evidenced that error distributions obtained with \emph{regseg} and
  the \gls*{t2b} were significantly different, and using the violin plots in box B of
  \autoref{fig:results_real} we observed that errors are always larger for \gls*{t2b}.
We also reported the 95\% CIs of the \gls*{swindex} for those surfaces.
The aggregate CI for \emph{regseg} was 1.08 - 1.50 [mm], whereas the \gls*{t2b} method
  yielded an aggregate CI of 2.06 - 2.43 [mm].
The results of the statistical tests and the CIs are summarized in \autoref{tab:results_real}.



\begin{table}
		\centering
		\footnotesize
		\tabcolsep=0.08cm
    \begin{tabular}{cccccc}
    & & $\Gamma_{VdGM}$  & $\Gamma_{WM}$ & $\Gamma_{pial}$ & Aggregate \\
    \hline
    \multirow{2}{*}{CI}
       & \emph{regseg}        & 0.50 - 0.78 & 0.50 - 0.55 & 0.66 - 0.73 & 0.56 - 0.66 \\
       & T2B                  & 1.78 - 2.58 & 1.94 - 2.36 & 2.16 - 2.58 & 2.05 - 2.39 \\
    \hline
    \multirow{2}{*}{H-tests}
       & p-value  & 4.1$\times$10$^{-6}$& 2.3$\times$10$^{-6}$& 2.3$\times$10$^{-6}$ & 1.8$\times$10$^{-16}$ \\
       & H-stat   & 21.20               & 22.31               & 22.31                & 67.85              \\
    \hline
    \end{tabular}
    \caption{The statistical analysis of results on real data supported that \emph{regseg} overperformed
    the alternative \acrfull{t2b} method.
    The distribution of errors computed for the surfaces of interest ($\Gamma_{VdGM}$, $\Gamma_{WM}$, $\Gamma_{pial}$)
      and the aggregate of all surfaces presented lower 95\% CIs for \emph{regseg}.
    CIs reported in this table were computed using bootstrapping with mean as location statistic and 10$^4$ samples.
    The Kruskal-Wallis H-tests indicated that there is a significant difference between \emph{regseg} and
      the \gls*{t2b} method.
    }\label{tab:results_real}
\end{table}

\begin{figure*}
    \centering
    \resizebox{\textwidth}{!}{%
      \begin{tikzpicture}
        \node[inner sep=15pt, draw=black!40](fig05a) at (0,11)
          {\hspace*{20pt}\includegraphics[width=\linewidth]{figures/figure05-a}};
        \node[inner sep=15pt, draw=black!40](fig05b) at (0,0)
          {\hspace*{20pt}\includegraphics[width=\linewidth]{figures/figure05-b}};
        % \node[inner sep=5pt, draw=black!40](fig03c) at (0,0)
        %   {\includegraphics[width=\linewidth]{figures/figure03-c}};
        \node[circle, text=black!75] at (-9.5,17) {\Huge \textbf{A}};
        \node[circle, text=black!75] at (-9.5,3) {\Huge \textbf{B}};
        % \node[circle, text=black!75] at (-8.8,3.4) {\Huge \textbf{C}};
      \end{tikzpicture}
    }%
	\caption{A. Example of one report for visual assessment, automatically generated by the evaluation instrument.
    Each view shows one component of the input image (in this case, the \gls*{fa} map), the ground-truth location
    of the surfaces (green contours), and the resulting surfaces with the method under test (yellow contours).
  First two rows show axial slices for \emph{regseg} and the \acrfull*{t2b} method, and the last two rows
    show corresponding sagittal views.
  Coronal view is omitted since it is the least informative due to the the directional property
    of distortions.
	Red arrows point to regions where \emph{regseg} overperformed the \gls*{t2b} method.
  B. Violin plots of error distributions of each surface, with indication of the voxel size of the \gls*{dmri} images
    (1.25 mm), and supporting the improved results of \emph{regseg} in the proposed settings.
	}\label{fig:results_real}
\end{figure*}