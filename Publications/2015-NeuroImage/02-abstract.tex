% -*- root: 00-main.tex -*-
Integrating information from different images is an ubiquitous necessity
  in current neuroimage analyses.
However, oftentimes one of the images is not available or the objective
  function is not adequate or numerically tractable.
We propose a surfaces-to-volume registration method called \emph{regseg}
  that exploits the detailed anatomy extracted from structural images and
  simultaneously segments the target image with subvoxel precision.
We use active contours without edges to search for a deformation field
  of B-Spline functions that optimally maps the surfaces
  onto the data in the target space.
We test \emph{regseg} on four digital phantoms warped with random deformations,
  and conclude that the precision of the method is below the half-voxel spacing.
Finally, we apply \emph{regseg} in the correction for susceptibility-derived
  distortions of 16 real diffusion MRI datasets from the \acrlong*{hcp}
  database.
The results suggest that \emph{regseg} overperformed an extended competing method.
\emph{Regseg} showed a consistent error within the 95\% confidence interval of
  $0.39-0.93mm$ in images of $1.25mm$ isotropic resolution.

