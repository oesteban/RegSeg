\documentclass[9pt]{memoir}

\usepackage[numbers,sort&compress]{natbib}
\usepackage{xcolor}
\usepackage[T1]{fontenc}
\usepackage{charter}

\usepackage{xcite}
\externalcitedocument{00-main}

\usepackage[hidelinks]{hyperref}
\usepackage{nameref}

\usepackage{amsmath}
\usepackage{amssymb}

\newcounter{reviewpoint}
\makeatletter
\newenvironment{reviewpoint}%
{\refstepcounter{reviewpoint}\par\medskip\vspace{3ex}\hrule\vspace{1.5ex}\par\noindent%
   {\fontseries{b}\selectfont Comment \arabic{reviewpoint}:}
   \begingroup%
   \color{black!60}
   \fontshape{it}\selectfont %

}
{\endgroup\label{com:\thereviewpoint}\par\medskip}
\def\reviewpointautorefname{Comment}
\makeatother


\newcommand{\reply}{\par\fontshape{n}\selectfont \noindent \textbf{Reply}:\ }

\setlrmarginsandblock{3cm}{2.5cm}{*}
\setulmarginsandblock{2.5cm}{2.5cm}{*}
\checkandfixthelayout

\begin{document}

\section*{Response letter to the manuscript no. NIMG-15-1151}
\medskip
Original title: \emph{Active contours-driven registration method for the structure-informed segmentation of
  diffusion MR images}.

\noindent New title: \emph{{\color{blue!70} Surface}-driven registration method for the structure-informed segmentation of
  diffusion MR images}.

\bigskip
\noindent We thank the editors and the reviewers for revising our manuscript no. NIMG-15-1151 submitted to NeuroImage.
We appreciate the comments made by both reviewers, and considered them fully to improve our paper.
Please, find in the following our point-by-point answer to the concerns raised by the reviewers,
  along with the description of the corresponding changes in the manuscript.
We have also changed the color of text for the modifications in the manuscript, along with side notes indicating which reviewer
  and comment suggested the edition.

\bigskip
\bigskip
\subsection*{Reviewer \#1:}
\begin{reviewpoint}
The paper describes an approach for the delineation of selected anatomical structures ([...]) in multi-shell diffusion MR images
  by registering a segmented T1 weighted reference image with fractional anisotropy and apparent diffusion coefficient maps derived
  from the diffusion weighted images.

The approach corresponds fully to the usual registration of an atlas to the image to be segmented, leaving me rather clueless why do
  they call the method ``simultaneous registration and segmentation''.

\end{reviewpoint}
\reply{%
We decided to call the method ``simultaneous registration and segmentation'' because antecedent methods
  using active contours that evolve driven by the optimization of a deformation field use that nomenclature.
These methods are now summarized and discussed in lines 234--252 of the current version.
}

\begin{reviewpoint}
The paper describes an approach [...] (quite some are listed as possible examples but results are only reported for the white matter,
  deep gray matter and the pial surface) in multi-shell [...].
\end{reviewpoint}
\reply{
We have defined clearly that the choice of surfaces to segment the target data is crucial, and define the
  segmentation model that drives the registration process (lines 85--88 are new).
We have also improved former Figure 4, and brought it earlier in the paper (now it is Figure 2).
Since the former visualization of the conditional (region-wise) joint distribution of FA and ADC was a bit clumsy,
  we have extracted the conditional distributions to the right side of each panel.
}

\begin{reviewpoint}
The proposed algorithm is rather standard, relying on a region-based elastic contour deformation approach.
The cost function is derived in the usual Bayesian framework, relying on a multivariate normal distribution prior for the image
  intensities and a Tikhonov regularizer enforcing a predefined normal distribution over the distortions along the imaging axes.
In this respect the novelty of the paper can be questioned, while the application context and the very promising results may still
  justify a publication in NeuroImage.
\end{reviewpoint}
\reply{
We agree with the reviewer on that the novelty of the methodological elements of our approach is limited.
However, to our knowledge there exists no alternative method combining the same elements to segment
  multivariate data.
We have moved the related methods and our methodological contributions from the Introduction to a more appropriate
  place in the Discussion (ll. 234--252), additionally satisfying \autoref{com:9} suggesting to include in the introduction
  only the references related to the focus of the paper.
}

\begin{reviewpoint}
However, in such a case the paper should be thoroughly revised to address the following major issues:
1. The assumption of independence between pixels is quite usual but obviously wrong.
The authors may want to comment on the consequences.
\end{reviewpoint}
\reply{
The reviewer is right, assuming that pixels are i.i.d. is wrong but a widely accepted simplification.
We have supported the assumptions inserting the new lines 80--83.}

\begin{reviewpoint}
2. It is unclear, under which conditions enforcing the optimality condition guarantees that the region identified corresponds
  to the targeted one (it probably strongly depends on its intensity properties and those of neighbouring structures).
A corresponding analysis of the goal function and the presence of (undesired) local minima is missing.
Figure S1 is not particularly helpful in this respect.
Some information about the initialization of the distortion map should also be given.
\end{reviewpoint}
\reply{
We have improved former Figure 4 (now, it is Figure 2) to clarify that the segmentation model is inherently defined by
  the choice of surfaces and their nesting.
This is an actual contribution of the work, as stated in ll. 264--266.

The model behaves robustly when a) the target data are simple as in our phantom settings; and b) when the distortion
  has only one nonzero component like in our real-data settings.
The analysis of the goal function for the phantoms (without directional restrictions) was not necessary since
  convergence was straightforward.
In the case of real data, since the distortion under investigation is aligned with one axis, the Figure S1 shows
  that the goal function is well defined.

However, the reviewer is right when raising convergence issues.
We noticed that, when decimating the meshes in the phantom experiments, the zero of shape-gradients moved away from
  the minimum of the goal function.
Since decimation factors applied in this experiment were very large, and it did not affect our results with
  the meshes directly obtained with \emph{Freesurfer} (in both phantoms and real data), we did include this discussion
  within the paper.

Finally, we commented on the initialization issue with the insertion of ll. 188--190.
}

\begin{reviewpoint}
3. I miss a reasonable description of the parameter settings necessary for generating the results presented.
It is not even clear if all experiments have been performed by the same parameter settings.
A reasonable sensitivity analysis is also missing.
\end{reviewpoint}
\reply{
We have made a strong effort in providing reproducible experiments using openly available data.
Therefore, parameter settings are publicly available in GitHub.
Particularly, under folder \url{https://github.com/oesteban/RegSeg/tree/master/Scripts/pyacwereg/data}
  there are several configuration files.
Some parameters are also discussed in the Supplemental Materials, Section S2.

However, we understand that a better description of the parameter settings should be included within
  the manuscript and modified it accordingly, in ll. 192--196.
}

\begin{reviewpoint}
4. I could not understand how the ground truth can be generated for the real datasets.
As far as I can see, in these cases there is a (probably unknown) distortion between the T1 weighted and the
  diffusion weighted images.
How an emulated distortion can get around this fact?
\end{reviewpoint}
\reply{
As commented in line 160, we are using some ``minimally preprocessed'' images from the database of the
  Human Connectome Project (HCP).
Particularly, the dMRI datasets from these preprocessed images are given corrected for distortions
  and registered in structural space.
Therefore, there could be a residual misregistration between the T1w and the dMRI data, but it should be
  very minimal.
We have modified former Figure 3 (now Fig. 4) to reflect this aspect.
}

\begin{reviewpoint}
5. The vertex-wise error distributions are not only skewed, but quite clearly structured im many cases
  (especially for the phantom images).
What is the underlying reason?
\end{reviewpoint}
\reply{
We have added a discussion on this in the corresponding section, lines 286-291. Thanks for pointing out.
}

\bigskip
\bigskip
\subsection*{Reviewer \#2:}
\begin{reviewpoint}
This paper seems to present a method that can perform wM-GM segmentation and simultaneously estimate a deformation field to
  correct for EPI distortion.
Apparently, the idea is to use nested WM-GM surfaces, as well as other surfaces, to provide better cortical registration than
  a competing method, which is not clearly described.
The overall goal and motivation of the paper are not clearly explained, and almost all related references for EPI distortion
  correction are missing.
It is unclear if the deformation field solution is restricted to the phase encode direction.
It is possible that this is a very useful method, as the results seem to indicate, but the paper is unfortunately too unclear to
  make a full judgement about the method.
\end{reviewpoint}
\reply{
We thank the reviewer for pointing out that the overall goal is not clearly explained.
We have replaced the term ``active-contours'' by ``surfaces'' whenever possible.
We have re-organized the introduction fully, as suggested (see reply to \autoref{com:1} for a brief explanation).
We have included references of the main EPI distortion corrections (see lines 31-43).

The \emph{regseg} tool is designed to allow any displacement, as demonstrated with the phantoms.
However, as the reviewer points out, the deformation field is completely restricted to the phase-encode direction
  for the real datasets.
We have added ll. 125--127 to indicate that the ground-truth in the experiments done with real data is directionally
  restricted. Then, in lines 190-192 to clarify that \emph{regseg} is configured to fulfill this feature.
}

\begin{reviewpoint}
abstract: Not immediately clear what this means: ``active contours without edges extracted from structural images''
  (Where do the edges come from? What modalities are actually input to regseg?)
\end{reviewpoint}
\reply{%
In our adaptation of ``active contours without edges'' \citep{chan_active_2001}, the active contours are
  ``extracted from structural images''.
Then, they are ``without edges'' because the target features do not show clear intensity steps
  where the contours are intended to be finally placed.
In practice, active contours are ``without edges'' when they look for homogeneous regions enclosed
  by the contours, as is the case of the FA and ADC.
We agree with the reviewer on that this phrase is not clear, and therefore reformulated the abstract.
We have also modified the introduction to state more clearly the actual modalities that input to \emph{regseg}.
}

\begin{reviewpoint}
I note that the text in the abstract seems to have no correspondence with the graphical abstract.
The text abstract mentions nothing about diffusion MRI, FA, ADC, or phase maps, so it is unclear how these may be used in the method.
These images figure prominently in the graphical abstract, which may be a better overview than the text abstract.
\end{reviewpoint}
\reply{We agree with the reviewer, we have updated the text abstract, and proposed a completely new graphical abstract.}

\begin{reviewpoint}
Intro line 6-7. I am sure you mean to say that the resolution of dMRI is much LOWER than the images microstructure (not higher).
  larger voxels == lower resolution
\end{reviewpoint}
\reply{Absolutely, we have fixed the manuscript accordingly.}

\begin{reviewpoint}
Intro 9. Orientations is the usual term for diffusion weighting, rather than ``angles''.
\end{reviewpoint}
\reply{We agreed, and updated the manuscript accordingly.}

\begin{reviewpoint}
``These limitations prevent segmentation in the native dMRI space''  I disagree.
In fact, many methods have been used to segment fiber tracts, then to quantitatively measure them,
  in the dMRI space.
It is true the correspondence with T1 is imperfect, but segmentation of fiber tracts is possible.
\end{reviewpoint}
\reply{
The reviewer is right, we have changed the paragraph about dMRI segmentation (lines 15-27) to
  be more precise.
We agree with the reviewer that there actually exist many methods to segment fiber tracts.
However, our method is intended to support fiber tracking methods requiring detailed anatomical
  information, such as \citep{smith_anatomicallyconstrained_2012}.
We have supported the applications more clearly on lines 11-14.
Clustering tractograms could be another potential application of \emph{regseg},
  but we agree with the reviewer that current methods perform well without
  additional structural information.
}

\begin{reviewpoint}
Please define what you mean here by segmentation.
It seems there is an assumption that it means specifically segmentation via registration, where the
  structures are defined using T1 images, but that is not stated anywhere.

Now on line 19 it seems the goal of the segmentation may simply be the segmentation of the WM.
So far, I have no idea what the overall goal of the approach is.
\end{reviewpoint}
\reply{
This comment is very related to \autoref{com:14}.
We have modified the abstract and introduction accordingly.
}

\begin{reviewpoint}
Line 24. clustering of what?
\end{reviewpoint}
\reply{
This comment is a consequence of the two previous comments.
The reviewer is right in the sense that we did not clearly state what we are segmenting, and therefore
  the references in line 24 are a bit orphaned.
All methods reviewed in lines 19-28 cluster/segment/classify \emph{voxels} in dMRI space based on
  one or another feature derived from dMRI data.
We have clarified the manuscript according to \autoref{com:14} and \autoref{com:15}, and specifically
  rewritten lines 19-28 to explicitly state what those methods are clustering.
}

\begin{reviewpoint}
Now in the paragraph beginning on line 29, it seems the segmentation of WM/GM interface is of interest.
Still, I have no idea if this is the goal of the approach, or why.

Line 39. I note that there are many approaches, in addition to nonlinear T2->b0, that have been proposed
  for correction of EPI distortions in dMRI.
If that is the goal of this approach, the entire intro should be rewritten focusing on this related work.
However, I do not know yet what the goal is.

The entire Introduction should be rewritten to improve organization and clearly state the paper's goal much
  earlier (in abstract and introduction).
Truly related work should be described in more detail, and all of the unrelated references should be removed.

Line 58. It seems the paper goal may be stated here. It is still completely unclear what type of segmentation should be achieved.
\end{reviewpoint}
\reply{
Again, the reviewer finds that the main goal of the paper is not clearly stated.
We understand that modifications derived from the previous comments satisfy the answer to the concern,
  and changes include abstract and introduction.
We have stated more clearly the importance of the set of surfaces being registered to produce a reliable
  segmentation model of the two diffusion features selected: FA and ADC
}

\begin{reviewpoint}
It seems by line 69 that EPI distortion correction and segmentation by registration are to be performed jointly
  in some fashion.
There are many, many references for distortion correction that are missing from this introduction and that must
  be added.
See for example multiple papers by Carlo Pierpaoli's group, including this most recent:

Irfanoglu, M. Okan, et al. ``DR-BUDDI (Diffeomorphic Registration for Blip-Up blip-Down Diffusion Imaging) method for correcting echo planar imaging distortions.''
  NeuroImage 106 (2015): 284-299.
\end{reviewpoint}
\reply{
We have added the suggested reference (among other of interest) in lines 204-206.
In lines 59-61 we have stated the differences of \emph{regseg} w.r.t. EPI distortion corrections.
}

\begin{reviewpoint}
The majority of the EPI distortion correction methods use T2 and B0 due to similar contrast. Why was T1 chosen in this framework?

Suddenly in Section 2 it becomes clear that there are surfaces being registered.
Perhaps this should have been obvious from the title, which mentioned active contours, but before line 72 it was not clearly
  stated that the method was posed as a surface registration problem.
\end{reviewpoint}
\reply{
We have updated the title to represent this aspect.
We have also included all previous changes to fulfill this request.
}

\begin{reviewpoint}
Was the warp restricted to the phase encode direction?
If so, the r in eq (1) should be restricted to this direction.
I do not see this mentioned anywhere near the equations in the Methods section.
If the solution found for the deformation field was not restricted to the phase encode direction,
  then this is not a reasonable way to perform EPI distortion correction. Please clarify.

[...]

I still see nothing in the results indicating that the estimated distortions were restricted to the phase encode direction.
If not, the added degrees of freedom might result in improved surface segmentation, but the distortion field would not correspond at
  all to the physical problem under study.
Furthermore, the dMRI gradients would have to be re-oriented independently at each voxel, and this is not currently possible in any
  software I know of.
\end{reviewpoint}
\reply{
The displacements field supporting the deformation model is not restricted to the phase encoding direction by default.
Therefore, equations in Methods section are presented under such conditions.
The proof of concept presented in the first experiment with phantoms are conducted without preferential directions
  for the deformation.
However, the point raised about the distortions derived from susceptibility inhomogeneity in dMRI data is also true.
Therefore, experiments with real data mimicking such distortions are performed with the deformation aligned with the phase encoding
  axis.
New lines 203-205 now state this point clearly.
}

\begin{reviewpoint}
The experiments seem reasonable and clearly presented in general. But far too little detail is given on the B0-T2 comparison method.
\end{reviewpoint}
\reply{
We thank the reviewer for positively weighing our experiments and results.
As stated in line 214, we use the settings from the \emph{ExploreDTI} software package \citep{leemans_exploredti_2009}, that
  internally uses \emph{elastix} \citep{klein_elastix_2010} in registration.
A footnote in line 215 directly links the settings for this method.
}

\begin{reviewpoint}
Please define the phrase ``active contours without edges''.
\end{reviewpoint}
\reply{We hope this point is answered in \autoref{com:10}, and tried to avoid this concept
  coming from the 2D processing field.
Now, we refer to the contours as surfaces whenever possible.
Regarding the ``without edges'' feature, we have clarified how the surfaces enclose homogeneous
  regions (and therefore, they are not used to search for image edges).
}

\begin{reviewpoint}
Figure 3: What are the arrows in images 5? DEformations? Indications of important regions?
\end{reviewpoint}
\reply{
We have modified the caption of former Figure 3 (now Fig. 4) to describe the meaning of the arrows correctly.
}

\bibliographystyle{mystyle}
\bibliography{Remote}

\end{document}
