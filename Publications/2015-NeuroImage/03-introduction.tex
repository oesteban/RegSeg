% -*- root: 00-main.tex -*-
\section{Introduction}%
\label{sec:introduction}
Image registration and segmentation are crucial in current pipelines for
  analysis of whole-brain connectivity using \gls*{dmri}, such as the extraction
  of brain connectivity \citep{craddock_imaging_2013}.

An early integration of segmentation and registration by \cite{bertalmio_morphing_2000} proposed
  a sequential deformation of active contours for object tracking in series of 2D images.
Shortly, \cite{yezzi_variational_2003} presented the first method including a full solution to
  the registration problem with an affine transformation supporting the mapping.
\cite{vemuri_joint_2003} proposed an atlas-based registration framework using level sets and only
  one \gls*{pde} for first time.
\cite{unal_coupled_2005} and later \cite{wang_joint_2006},
  extended the ``two \glspl*{pde}'' method of \cite{yezzi_variational_2003}
  to nonlinear registration implementing a free deformation field.
\cite{droske_mumfordshah_2009} reviewed the latter set of techniques, and proposed two different
  approaches to apply the Mumford-Shah functional \citep{mumford_optimal_1989} in simultaneous
  registration and segmentation, through the propagation of the deformation field from
  the contours to the whole image definition.
\cite{greve_accurate_2009} presented a widely used registration method called \emph{bbregister},
  included in \emph{FreeSurfer} \citep{fischl_freesurfer_2012}.
Their framework performs robust registration of brain surfaces into the intensity information
  of a target \gls*{mri}, using an affine transformation and active contours.
Recently, \cite{guyader_combined_2011} proposed a simultaneous segmentation and
  registration method using level sets and a nonlinear elasticity smoother on the
  displacement vector field, which preserves topology even with very large deformations.
Finally, \cite{gorthi_active_2011} extended the existing methodologies using a multiphase
  level-set function for the registration of several active contours, in the application
  of atlas-based segmentation.
Alternatively to the historical use of active contours, some Bayesian approaches
  have been proposed as well \citep{wyatt_map_2003,pohl_bayesian_2006,gass_simultaneous_2014}.

We propose \emph{regseg} as a solution to joint segmentation and registration 
  problems with a shape priors approach.
An immediate application of \emph{regseg} is the correction of susceptibility-derived
  distortions typically present in \gls*{dmri} data and its segmentation
  after correction.
Both correction and segmentation are required steps in current investigations of whole-brain
  tractography, such as the connectome extraction \citep{daducci_connectome_2012}.
Indeed, the joint segmentation and correction can exploit the anatomy extracted from
  the \gls*{t1} as strong shape-priors.