\section{INTRODUCTION}

In-vivo whole-brain connectivity analysis has been a
research topic of high interest for the last
5 years \cite{jones_studying_2008}. 
\Gls*{dmri} can be used to probe the
orientation of fiber bundles within the brain,
generally applying \gls*{epi} sequences.
After a signal reconstruction step 
(i.e. \cite{basser_microstructural_2011}), 
tractography algorithms draw a map of the sampled 
structures.
These maps can represent the actual trajectories
of fiber bundles (deterministic tractography) or
pixel-wise probability of connection to a certain origin
(probabilistic tractography). Finally, the
information about these connections is collected
into a network matrix that can be subjected to
the so-called ``connectome analysis''.

However, such a complex workflow is prone to
flaws that have been detailedly reported 
\cite{jones_twenty-five_2010,jbabdi_tractography:_2011}.
In order to enable reliable and robust connectome
analyses of the brain, some weak processing points must
be improved. One prominent source
of inaccuracy are the susceptibility-derived artifacts,
for which \Gls*{epi} schemes are specially sensitive.
As susceptibility changes at tissue interfaces,
so does the magnetic field. The consequence of the 
inhomogeneity of the field translates in a highly 
distorted imaged anatomy and a significant signal 
destruction of certain regions of the brain 
(e.g. the orbitofrontal lobe, for the proximity of the
air surrounding sinuses). This artifact has been
profoundly described, generally within \gls*{fmri}
literature because they are usually acquired with 
\gls*{epi} as \gls*{dmri}. 

First approaches to
retrospective correction \cite{jezzard_correction_1995},
relied on an extra \gls*{mri} acquisition (so-called
field mapping), that probes the inhomogeneity of the field.
A second theory-based breed of methodologies acquire a 
map of the \gls*{psf} of the \gls*{epi} readouts to correct 
for the artifact \cite{robson_measurement_1997}. 
Next generation of methodologies \cite{cordes_geometric_2000,
chiou_simple_2000}, make use of the acquisition of 
extra \gls*{epi} volumes 
with specific differences that enable correction, for instance
the opposed direction of the phase encoding gradients 
Finally, the last family of methodologies acquire an 
extra \gls*{t2} image, and use its anatomical correctness 
as reference to find the deformation map through nonlinear 
registration \cite{kybic_unwarping_2000,studholme_accurate_2000}.
Registration-based methods usually map
the \gls*{t2} image to the so-called baseline (B0) volume 
of \gls*{dmri}. The choice of \gls*{t2} is due to the strong
similarity of intensity distribution with the B0 of \gls*{dmri}
datasets. More recent works report extensions of the existing
techniques \cite{andersson_how_2003,holland_efficient_2010,
andersson_comprehensive_2012}, and combined approaches
\cite{zaitsev_point_2004}.

Even though aforementioned techniques for distortion correction
have been studied \cite{zeng_image_2002,wu_comparison_2008},
the lack of a gold-standard limits the possible benchmarking 
strategies. \cite{irfanoglu_effects_2012} raised the question
of distortion-derived impacts in tractography.
In this work, we propose an evaluation framework
using a detailed digital phantom designed for tractography.
Using this framework,
we are able to compare several correction techniques and we
provide results related to their performance in the task
itself and their impact on subsequent tractography.