\section{METHODS}

\subsection{Digital Phantom}
We modified the digital phantom provided with the
\emph{\gls*{hardi} reconstruction challenge} held
in ISBI 2013, San Francisco, US. This model is
built upon high-resolution \glspl*{tpm}, from which
\gls*{dmri} signal can be simulated with different
\glspl*{snr}. The new version of the phantom includes
a \gls*{csf} distribution, which was not simulated
in the version available for the contest. For this paper,
with these new \glspl*{tpm} we produced \gls*{dti} at
typical low-resolution, and corresponding \gls*{t1} and 
\gls*{t2} images at high-resolution. All the
modalities where numerically simulated, using previously
reported \gls*{mri} typical parameters.


\begin{table}[h]
\caption{Simulated \gls*{mri} modalities}
\label{table:phantom}
\begin{center}
\begin{tabular}{c|c}
\hline
Modality & Descrition \\
\hline
\gls*{dti} & 30 directions, \\
\hline
\gls*{t1} & \\
\hline
\gls*{t2} & \\
\hline
\end{tabular}
\end{center}
\end{table}


\subsection{Theory-based synthetic distortion}

   \begin{figure}[thpb]
      \centering
      \framebox{\parbox{3in}{Figure of the phantom}}
      %\includegraphics[scale=1.0]{figurefile}
      \caption{Original and distorted digital phantom}
      \label{fig:label}
   \end{figure}
   

\subsection{Evaluation workflows}


